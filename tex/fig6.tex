\documentclass[11pt]{article}


\usepackage{amssymb,amsfonts}
\usepackage[figuresright]{rotating}
\usepackage{amsmath,amssymb}
\usepackage{graphicx}% Include figure files
\usepackage{dcolumn}% Align table columns on decimal point
\usepackage{bm}% bold math
\usepackage{amscd,amsthm}
\usepackage{ifthen}
\usepackage{mathtools}

\usepackage{pgf,tikz}
\usepackage{pgfplots}
\usetikzlibrary{calc,shadows}
\usetikzlibrary{pgfplots.groupplots}
\usepackage{subfigure}
\usepackage{url}
\usetikzlibrary{pgfplots.groupplots}

%\usetikzlibrary{external}
%\tikzexternalize[prefix=./]
%\tikzset{external/force remake}

\renewcommand{\S}{S}
\renewcommand{\L}{{L}} 
\newcommand{\N}{{N}} 
\newcommand{\TL}[1]{\tilde{#1}} 
\newcommand{\vy}{\mathbf y}
\newcommand{\vn}{\mathbf n}
\newcommand{\complexset}{\mathbb C}



\usetikzlibrary{pgfplots.groupplots}
\usetikzlibrary{matrix,arrows,decorations.pathmorphing}
\usepackage{pgfplots,pgfplotstable}
\usepgfplotslibrary{fillbetween}
\pgfplotsset{compat=1.10}
\usetikzlibrary{matrix,arrows,decorations.pathmorphing}

\newcommand{\errorband}[6]{ \pgfplotstableread{#1}\datatable \addplot
  [name path=pluserror,draw=none,no markers,forget plot] table
  [x={#2},y expr=\thisrow{#3}+\thisrow{#4}] {\datatable};

  \addplot [name path=minuserror,draw=none,no markers,forget plot]
    table [x={#2},y expr=\thisrow{#3}-\thisrow{#4}] {\datatable};

  \addplot [forget plot,fill=#5,opacity=#6]
    fill between[on layer={},of=pluserror and minuserror];

  \addplot [#5,thick,no markers]
    table [x={#2},y={#3}] {\datatable};
}

\begin{document}


\begin{figure}[h!]
\begin{center}
\begin{tikzpicture}
\begin{groupplot}[
         title style={at={(0.5,-0.3)},anchor=north}, group
         style={group size=2 by 1, horizontal sep=2cm },
         width=0.5\textwidth, xticklabel style={/pgf/number
           format/fixed, /pgf/number format/precision=3}, yticklabel
         style={/pgf/number format/fixed, /pgf/number
           format/precision=5}, ymin = 0] % left panel
  \nextgroupplot[xlabel = {$\lambda$},ylabel = {relative sample
      complexity},title={(a)}]
  \errorband{../fig/generalization_n10.dat}{0}{3}{4}{blue}{0.2}
  \addlegendentry{AR} %
  \errorband{../fig/generalization_n10.dat}{0}{5}{6}{red}{0.2}
  \addlegendentry{PLPAC} %
  \errorband{../fig/generalization_n10.dat}{0}{7}{8}{brown}{0.2}
  \addlegendentry{BTMB}
        
	% right panel 
\nextgroupplot[xlabel = {$\lambda$},ylabel =
  {failure probability},title={(b)}] \addplot +[mark=x] table[x
      index=0,y index=9]{../fig/generalization_n10.dat};
    \addlegendentry{AR} % 
\addplot +[mark=x] table[x index=0,y
    index=10]{../fig/generalization_n10.dat}; \addlegendentry{PLPAC} %
      \addplot +[mark=x] table[x index=0,y
        index=11]{../fig/generalization_n10.dat}; \addlegendentry{BTMB}
	 
\end{groupplot}          
\end{tikzpicture}
\end{center}
%
\caption{
%\label{fig:violationBTL} (a) Relative sample complexity
%  defined as the number of comparisons until termination, $\numcmp$,
%  divided by the complexity parameter $\complexityP(\score(\Pmat))$,
%  and (b) failure probability on a BTL model $\pmat$ with
%  $\numitems=10$ and with a fraction of $\lambda$ of the off-diagonals
%  of $\pmat$ substituted by a random pairwise comparison probability.
%  The model transitions from a BTL model to a random pairwise
%  comparison matrix in $\lambda$; the closer $\lambda$ to zero the
%  closer $\pmat$ to the original BTL model.  The results show that,
%  while the AR algorithm yields an $\delta$-accurate ranking after
%  $O(\complexityP(\score(\Pmat)))$ comparisons, irrespectively of
%  $\lambda$, the sample complexity and more importantly the failure
%  probability of the PLPAC and BTMB algorithms become very large in
%  $\lambda$. 
   }
\end{figure}


\end{document}
