\documentclass[11pt]{article}


\usepackage{amssymb,amsfonts}
\usepackage[figuresright]{rotating}
\usepackage{amsmath,amssymb}
\usepackage{graphicx}% Include figure files
\usepackage{dcolumn}% Align table columns on decimal point
\usepackage{bm}% bold math
\usepackage{amscd,amsthm}
\usepackage{ifthen}
\usepackage{mathtools}

\usepackage{pgf,tikz}
\usepackage{pgfplots}
\usetikzlibrary{calc,shadows}
\usetikzlibrary{pgfplots.groupplots}
\usepackage{subfigure}
\usepackage{url}
\usetikzlibrary{pgfplots.groupplots}

%\usetikzlibrary{external}
%\tikzexternalize[prefix=./]
%\tikzset{external/force remake}

\renewcommand{\S}{S}
\renewcommand{\L}{{L}} 
\newcommand{\N}{{N}} 
\newcommand{\TL}[1]{\tilde{#1}} 
\newcommand{\vy}{\mathbf y}
\newcommand{\vn}{\mathbf n}
\newcommand{\complexset}{\mathbb C}



\usetikzlibrary{pgfplots.groupplots}
\usetikzlibrary{matrix,arrows,decorations.pathmorphing}
\usepackage{pgfplots,pgfplotstable}
\usepgfplotslibrary{fillbetween}
\pgfplotsset{compat=1.10}
\usetikzlibrary{matrix,arrows,decorations.pathmorphing}



\begin{document}


\begin{figure}[h!]
\begin{center}
%\pgfplotsset{yticklabel style={
%        /pgf/number format/fixed,
%        /pgf/number format/precision=1
%},
%scaled y ticks=false}
\begin{tikzpicture}[scale=1.2]
\begin{groupplot}[
         title style={at={(0.5,-0.25)},anchor=north}, group
         style={group size=2 by 1, horizontal sep=1.5cm },
         width=0.45\textwidth, xticklabel style={/pgf/number
           format/fixed, /pgf/number format/precision=3}, yticklabel
         style={/pgf/number format/fixed, /pgf/number format/precision=2 }, %ymode=log, 
           ] % left panel 


\nextgroupplot[xlabel = {$n$}, ylabel = {sample complexity},title={(a)},legend style={at={(0.4,0.9)}}]
\addplot [no markers,blue,very thick] table [x index=0,y index=3] {../fig/comparison_varyn09.dat};     
\addlegendentry{AR}

\addplot [no markers,red,densely dotted,very thick] table [x index=0,y index=4] {../fig/comparison_varyn09.dat};     
\addlegendentry{PLPAC}

\addplot [brown,dashed,very thick] table [x index=0,y index=5] {../fig/comparison_varyn09.dat};     
\addlegendentry{BTMB}


%%%

%\nextgroupplot[xlabel = {$\pmatmax$},%ymax=300000,
%  %ylabel = {sample complexity},
%  title={(b)},ymode=log,
%  legend style={at={(1,1)}},
%  ]
%      \errorband{./fig/comparison_vary_closeness_linsep_rev2.dat}{1}{3}{4}{brown}{0.2}{dashed}
%      \addlegendentry{PBR} %
%\errorband{./fig/comparison_vary_closeness_linsep_rev.dat}{1}{6}{7}{red}{0.2}{densely dotted}
%   \addlegendentry{SAVAGE} %   
%  \errorband{./fig/comparison_vary_closeness_linsep_rev.dat}{1}{3}{4}{blue}{0.2}{solid}
%\addlegendentry{AR} %

%%%


\end{groupplot}     
\end{tikzpicture}
\end{center}
%\caption{\label{fig:varyn}
%(a) 
%Empirical sample complexity of the AR, PLPAC, and BTMB algorithms applied to the BTL model $\pmat^{(\xi)}$ with parameters $\parw_i = \xi (\numitems-i),
%  i=1,\ldots,10$ chosen such that $\pmatmax = 0.9$. 
%  Even in this regime where the PLPAC algorithm has a lower sample complexity than the AR algorithm, the difference is minor (30\% improvement at the most).
%(b) 
%Empirical sample complexity 
%of the PBR, SAVAGE, and AR algorithms 
%%of the PBR,  AR, PLPAC, and BTMB algorithms 
%applied to the BTL model $\pmat^{(\eta)}$ with parameters $\parw_i = \log(\eta + \numitems -
%  i), i = 1,\ldots,10$, as a function of $\pmatmax \defeq \max_{i,j} \pmat_{ij}$. We varied $\eta$ such
%  that $\pmatmax \in [0.65,0.99]$. The error bars correspond to one
%  standard deviation from the mean. 
%}
\end{figure}


\end{document}
